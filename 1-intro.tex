\section*{Overview}

\begin{frame}
\begin{itemize}
  \frametitle{Overview}
  \item The Problem: Power Law Graphs are Common
  \begin{itemize}
    \item An imporant class of natural graphs.
    \item A few \textit{very high-degree vertices}.
    \item Hard to partition.
    \item These vertices cause performance and scalability challenges for
          existing graph-parallel systems.
  \end{itemize}
\end{itemize}
\end{frame}

\begin{frame}
\begin{itemize}
  \frametitle{Overview}
  \item The Approach: The PowerGraph Abstraction
  \begin{itemize}
    \item \textbf{GAS}: A new 3-phase vertex-program methodology.
    \item ``Think like a vertex.'' \citep[SIGMOD '10]{malewicz2010pregel}
    \item Graph partitioning via vertex-cut, \textit{not} edge-cut (3 variants).
    \item Three execution modes (varying guarantees).
  \end{itemize}

  \item Evaluation:
  \begin{itemize}
    \item Comparison between 3 V-cut graph partitioning methods.
    \item Comparison between 3 execution modes.
  \end{itemize}
\end{itemize}
\end{frame}

\begin{frame}
\begin{itemize}
  \frametitle{Overview}
  \item Benefits:
  \begin{itemize}
    \item Vertex-oriented graph programming (``Think Like A Vertex'').
    \item Handles Large Power Law Graphs.
    \item Handles Very High-Degree Vertices.
    \item Scalable.
    \item Distributable/Parallelizable.
    \item Fault Tolerant.
  \end{itemize}
\end{itemize}
\end{frame}

\begin{frame}
\frametitle{Some Context\ldots}
\begin{itemize}
  \item \cite[OSDI '12 Paper]{gonzalez2012powergraph}
  \item \cite[OSDI '12 Video]{gonzalez2012powergraph-video}
  \item Work connected with larger GraphLab project.
  \item Work primarily done at CMU, but also UW.
  \item Tech commercialized by Dato, Inc. (GraphLab, Inc.)
  \item Current open sourced version of tech: SGraph
\end{itemize}
\end{frame}
