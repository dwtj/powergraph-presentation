\section*{Summary}

\begin{frame}{Related Work}
  \begin{itemize}
    \item \textbf{TODO:} Talk about related work here
  \end{itemize}
\end{frame}


\begin{frame}{Future Work}
  \begin{itemize}
    \item \textbf{TODO:} Talk about future work/limitations here
  \end{itemize}
\end{frame}


\begin{frame}
\begin{itemize}
  \frametitle{Overview}
  \item The Problem: Power Law Graphs are Common
  \begin{itemize}
    \item An imporant class of natural graphs.
    \item A few \textit{very high-degree vertices}.
    \item Hard to partition.
    \item These vertices cause performance and scalability challenges for
          existing graph-parallel systems.
  \end{itemize}
\end{itemize}
\end{frame}

\begin{frame}
\begin{itemize}
  \frametitle{Overview}
  \item The Approach: The PowerGraph Abstraction
  \begin{itemize}
    \item \textbf{GAS}: A new 3-phase vertex-program methodology.
    \item ``Think like a vertex.'' \citep[SIGMOD '10]{malewicz2010pregel}
    \item Graph partitioning via vertex-cut, \textit{not} edge-cut (3 variants).
    \item Three execution modes (varying guarantees).
  \end{itemize}

  \item Evaluation:
  \begin{itemize}
    \item Evaluate three V-cut graph partitioning methods.
    \item Evaluate three execution modes.
    \item Compare with other MLDM systems.
  \end{itemize}
\end{itemize}
\end{frame}

\begin{frame}
\begin{beamercolorbox}[center]{white}
  {\Large Questions?}

  \vspace{2em}\hfill

  \url{http://www.cs.iastate.edu/~dwtj}
\end{beamercolorbox}
\end{frame}
